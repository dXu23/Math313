\documentclass{article}
\usepackage[retainorgcmds]{IEEEtrantools}
\usepackage{amsmath}
\usepackage{amssymb}
\usepackage{enumerate}
\author{Daniel Xu}
\title{MATH 313 HMWK 9}
\begin{document}
\maketitle
\textbf{Exercise 4.4.7}
For any \(\epsilon\) given, we simply choose \(\delta = \epsilon \cdot \left|\sqrt{x} + \sqrt{x'}\right|\).
Then
\begin{IEEEeqnarray*}{r C l}
  \left| x - x' \right| & < & \epsilon \cdot \left|\sqrt{x} + \sqrt{x'} \right| \\
  \frac{ \left|x - x' \right| }{ \left| \sqrt{x} + \sqrt{x'} \right| } & < & \epsilon \\
  \left| \sqrt{x} - \sqrt{x'} \right| & < & \epsilon
  \end{IEEEeqnarray*}
\textbf{Exercise 4.4.9}
\begin{enumerate}[(a)]
\item
For any \(\epsilon\) given, we simply choose \(\delta = \epsilon / M\). Then
\begin{IEEEeqnarray*}{r C l}
  \left| x - y \right| & < & \frac{\epsilon}{M} \\
  M \cdot \left| x - y \right| & < & \epsilon
\end{IEEEeqnarray*}
We know that
\begin{IEEEeqnarray*}{r C l}
  \left| \frac{f(x) - f(y)}{x - y} \right| & \leq & M \\
  \left| f(x) - f(y) \right| & \leq & M \cdot \left| x - y \right|
\end{IEEEeqnarray*}
so
\begin{IEEEeqnarray*}{r C l}
  \left| f(x) - f(y) \right| & \leq & M \cdot \left| x - y \right| < \epsilon \\
  \left| f(x) - f(y) \right| & < & \epsilon
\end{IEEEeqnarray*}

\item False. The function \(\sqrt{x}\) is a counterexample. It is continuous on
  the compact set \([0, 1]\), yet it is not a Lipschitz function. For any bound \(M\) given,
  we can choose \(x = 1 / (M + 1)^{2} \) and \(y = 0 \) that do not satisfy the
  requirement of a Lipschitz function, as seen below.
  \begin{IEEEeqnarray*}{r C l}
    \left| \frac{f(x) - f(y)}{x - y} \right| & = & \frac{\sqrt{\dfrac{1}{(M + 1)^{2}}} - \sqrt{0}}{\dfrac{1}{(M + 1)^{2}} - 0} \\
      & = & \frac{1}{\left| M + 1 \right|} \cdot (M + 1)^{2} \\
      & = & M + 1 > M
  \end{IEEEeqnarray*}


\end{enumerate}
\textbf{Exercise 4.4.13}
For any \(\delta > 0\), we know that there exists \(N \in \mathbb{N}\) such that for any
\(m \geq N\) and \(n \geq N\), \(\left| x_{n} - x_{m} \right| < \delta\) for any Cauchy
sequence \(x_{n} \subset A\). By the definition of a uniformly continuous function, this
implies that \(\left| f\left( x_{n} \right) - f \left( x_{m} \right) \right| < \epsilon\),
which meets the definition of a Cauchy sequence.

\textbf{Exercise 4.5.2}
\begin{enumerate}[(a)]
\item Let us define the function \(f(x) = x^{3} - 9x\) on the open interval
  \((-3, 3)\). Then it will have range \([-6 \sqrt{3}, 6 \sqrt{3}]\), which
  is a closed interval. It should be readily apparent that for any \(x \in (-3, 3)\),
  \(\lim_{x \rightarrow c} f(x) = f(c)\), so \(f\) is continuous. 
\item Impossible. Let \(f\) be a function defined on the closed interval \([a, b]\)
  with range equal to the open interval \((c, d)\). Let us assume that \(f\) is a
  continuous function. Then for all \(V_{\epsilon} (a)\), there exists a \(V_{\delta} (c)\)
  with the property that \(x
\item Let \(f(x) = 1 / (1 - x^{2}\) be defined on the interval \((-1, 1)\).
  \item Impossible. 
  
\end{enumerate}

\textbf{Exercise 4.5.8}
We proceed by contradiction; we assume that \(f^{-1}\) is actually not
continuous. Then by \textbf{Corollary 4.3.3}, there exists a sequence
\((x_{n})\) in the domain of \(f^{-1}\) and a number \(c\) in the domain
of \(f^{-1}\), with \((x_{n}) \rightarrow c\) but \(f^{-1} (x_{n})\) does not
converge to \(f^{-1} (c)\). By \textbf{Theorem 4.3.2}, however, a characteristic
of our continuous function \(f\) is that
\(f^{-1} (x_{n}) \rightarrow f^{-1} (c) \implies (x_{n}) \rightarrow c\).
Since \(f^{-1} (x_{n}) \rightarrow f^{-1} (c)\) is false by \textbf{Corollary 4.3.3},
but \(f^{-1} (x_{n})\) is true, our implication is false; however, we know that
\(f\) is continuous. Thus, we must conclude that \(f^{-1}\) is also continuous. 

\end{document}
