\documentclass{article}
\usepackage{amsmath}
\usepackage{amssymb}
\usepackage[retainorgcmds]{IEEEtrantools}
\usepackage{enumerate}
\author{Daniel Xu}
\title{MATH 313 HMWK 7}
\begin{document}
\maketitle
\textbf{Exercise 3.2.2}
\begin{enumerate}[(a)]
  \item
  The limit points for \(A\) are 1 and -1. To see why, simply let
  \(a_{n} = A_{2n}\) and \(b_{n} = A_{2n - 1}\), which are sequences contained
  in \(A\). Then \(a_{n} = 1 + 1 / n\) and \(b_{n} = -1 + 2 / (2 n - 1)\). 
  We can use the Algebraic limit theorem to show that \(a_{n}\) converges
  to 1:
  \begin{IEEEeqnarray*}{r C l"s}
    \lim \left(1 + \frac{1}{n} \right) & = & \lim (1) + \lim \left(\frac{1}{n}\right) & by (ii) of Theorem 2.3.3 (Algebraic Limit Theorem) \\
    & = & 1 + 0 \\
    & = & 1
  \end{IEEEeqnarray*}
  We can show that \(b_{n}\) converges to -1 through the definition of the limit.
  For any arbitrary \(\epsilon > 0\), choose \(N > 1 / \epsilon + 1 / 2\). Then
  for all \(n \geq N\),
  \begin{IEEEeqnarray*}{r C l"s}
    n & > & \frac{1}{\epsilon} + \frac{1}{2} \\
    2n & > & \frac{2}{\epsilon} + 1 \\
    2n - 1 & > & \frac{2}{\epsilon} \\
    \left|\frac{1}{2n - 1} \right| & < & \frac{\epsilon}{2} & since \(n\) and \(\epsilon\) are both positive \\
    \left|\frac{2}{2n - 1} \right| & < & \epsilon \\
    \left| -1 + \frac{2}{2n - 1} - (-1)\right| & < & \epsilon \\
    \left| b_{n} - (-1) \right| & < & \epsilon
  \end{IEEEeqnarray*}
  Therefore, \(b_{n}\) converges to -1, so -1 is a limit point of \(A\).
  The limit points for \(B\) are all real numbers between 1 and 0 inclusive. By
  \textbf{Theorem 1.4.3}, for every real number \(r\) such that \(0 < r < 1\),
  there exists a rational number within \(V_{\epsilon}(r)\).
  
\item Set \(A\) is neither open nor closed. We choose the element \(2 = (-1)^{2} + (2 / 2)\) from the
  set \(A\), which we must show does not contain any \(\epsilon\)-neighborhoods that
  are a subset of \(A\). For 2 to contain a \(\epsilon\)-neighborhood, there must be
  an element \(a \in A\) such that \(a > 2\). However, it is not to hard to see that
  there is no such \(a\). Consider the sequences we defined before, \(a_{n}\) and
  \(b_{n}\). If we defined two sets to contain the elements of \(a_{n}\) and \(b_{n}\),
  and took the union of those sets, we would get \(A\). So if we can show that \(a_{n}\)
  does not contain any elements greater than 2 and \(b_{n}\) does not contain any elements
  greater than 2, then we can conclude that 2 does not have any \(\epsilon\)-neighborhoods
  that a subset of \(A\). So we assume to the contrary that \(a_{n}\) does contain an
  element greater than 2. Then for some \(n\), \(1 + 1 / n > 2\). However,
  \begin{IEEEeqnarray*}{r C l}
    1 + \frac{1}{n} & > & 2 \\
    \frac{1}{n} & > & 1 \\
    n & < & 1
  \end{IEEEeqnarray*}
  However, \(n\) must be a natural number, so it cannot be less than 1. Now we assume that
  \(b_{n}\) contains some element greater than 2. Then there exists some \(n\) such that
  \(-1 + 2 / (2n - 1) > 2\). However,
  \begin{IEEEeqnarray*}{r C l}
    -1 + \frac{2}{2n - 1} & > & 2 \\
    \frac{2}{2n - 1} & > & 3 \\
    \frac{2n - 1}{2} & < & \frac{1}{3} \\
    2n - 1 & < & \frac{2}{3} \\
    2n & < & \frac{5}{3} \\
    n & < & \frac{5}{6}
  \end{IEEEeqnarray*}
  As stated before, \(n\) must be a natural number, so \(n\) cannot be less
  than \(5 / 6\). Thus, \(b_{n}\) does not contain an element greater than 2.
  So, if neither \(a_{n}\) and \(b_{n}\) does not contain an element greater
  than 2, then \(A\) does not contain an element greater than 2; therefore,
  there is no \(\epsilon\) neighborhood for 2 that is a subset of \(A\) and
  \(A\) is not open.

  To show that \(A\) is closed, we just have to show that there is a limit point
  that \(A\) does not contain. We consider -1. In order for \(A\) to contain
  -1, there must be an element in \(A\) for which \(-1 = (-1)^{n} + 2 / n\).
  However,
  \begin{IEEEeqnarray*}{r C l}
    (-1)^{n} + \frac{2}{n} & = & -1 \\
    \frac{2}{n} & = & -1 - (-1)^{n} \\
    \frac{2}{n} & = & -2 \textrm{ or } 0
  \end{IEEEeqnarray*}
  Then \(n\) be equal to -1. However, \(n\) cannot be equal to -1, since \(n\)
  must be a natural number. And \(2 / n\) is never equal to 0. Therefore,
  the set \(A\) is closed.

  The set \(B\) is neither open nor closed. Any element \(b \in B\) will have
  a \(\epsilon\)-neighborhood containing a real number not part of the rational
  numbers, so for any \(V_{\epsilon}(b)\), \(V_{\epsilon}(b) \not\subset B\). Therefore,
  \(B\) is not closed. \(B\) is also not closed, since it does not contain its limit
  point 1. 

\item \(A\) does contain isolated points. Simply choose \(a \in A\) such that \(a \neq 1\)
  and \(a \neq -1\). \(B\), on the other hand, does not contain any isolated points.
  Since the limit points of \(B\) are all reals between 0 and 1 inclusive, \(B\) contains
  all of its limit points.
  
\end{enumerate}
\textbf{Exercise 3.2.3}
\begin{enumerate}
\item Open but not closed. By \textbf{Theorem 3.2.5}, \(e\) is a limit point of \(\mathbb{Q}\)
  since the sequence \(a_{n} = (1 + (1 / n))^{n}\) is contained in \(\mathbb{Q}\)
  due to the fact that rational numbers are closed under multiplication and
  addition. \(a_{n}\) converges to \(e\) which is an irrational number. 
  
\item Closed but not open. Choose any element \(n \in \mathbb{N}\). Then any \(V_{\epsilon} (n)\)
  will contain a number that is not in \(\mathbb{N}\). 

\item Open but not closed. 0 is most certainly a limit point of \(\{x \in \mathbb{R} : x \neq 0\}\),
  but is not in the set.
  
\item Not open and not closed. 1 is in the set, yet there is no \(\epsilon\)-neighborhood
  around 1 that is a subset of the set, since the \(\epsilon\)-neighborhood that will
  contain a real number greater than 1 but less than \(5 / 4\). Such a number will not be in
  the set, so for any \(\epsilon\), the \(\epsilon\)-neighborhood will not be a subset of the
  set. \(\pi^{2} / 6\) is a limit point of the set, yet the set does not contain \(\pi^{2} / 6\). 

\item Not open but closed. 1 is in the set, yet there is no \(\epsilon\)-neighborhood
  around 1 that is a subset of the set, since the \(\epsilon\)-neighborhood that will
  contain a real number greater than 1 but less than \(3 / 2\). Such a number will not be in
  the set. 

\end{enumerate}
\textbf{Exercise 3.2.6}
\begin{enumerate}
\item True. For any point in 
\item True. Since all closed intervals are closed sets, the Nested Interval Property
  still holds. 
\item True

  
\end{enumerate}

\textbf{Exercise 3.3.1}
By \textbf{Theorem 3.3.4}, any compact set will be closed and bounded. By the Axiom
of Completeness, any set that is bounded and is  a subset of the real numbers will have a least
upper bound and a least upper bound.

\textbf{Exercise 3.3.2}
\begin{enumerate}[(a)]
\item Not compact. Let \(a_{n} = n\), which we know to not contain any convergent subsequences. 
\item Compact.
\item
\item Not compact. We showed that this set was not closed since it does not contain
  the limit point \(\pi^{2} / 6\).
  \item Compact
\end{enumerate}

\textbf{Exercise 3.3.5}
\begin{enumerate}[(a)]
\item True. We consider \(n\) compact sets. By \textbf{Theorem 3.3.4}, any compact set is closed and bounded. By
  \textbf{Theorem 3.2.14}, the intersection of our compact sets will be closed.
  Since compact sets are bounded, there must exist \(M_{1}, M_{2}, \ldots M_{n}\)
  that are bounds on our \(n\) compact sets. For the intersection, we simply select
  bound \(M = \max\{M_{1}, M_{2}, \ldots M_{n}\}\). Since the intersection is bounded
  and closed, then it must be compact.
  
  \item True. We consider \(n\) compact sets. By \textbf{Theorem 3.3.4}, any compact set is closed and bounded. By
  \textbf{Theorem 3.2.14}, the union of our compact sets will be closed.
  Since compact sets are bounded, there must exist \(M_{1}, M_{2}, \ldots M_{n}\)
  that are bounds on our \(n\) compact sets. For the intersection, we simply select
  bound \(M = \max\{M_{1}, M_{2}, \ldots M_{n}\}\). Since the union is bounded
  and closed, then it must be compact.

\item False. Let \(A = (0, 1]\) and \(K = [0, 1]\). Then \(A \cap K = (0, 1]\),
    which is not compact since \(A \cap K\) is not closed.

    \item True by Nested Interval property. 
  \end{enumerate}

\end{document}
