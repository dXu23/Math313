\documentclass{article}
\usepackage{amsmath}
\usepackage{amssymb}
\usepackage[retainorgcmds]{IEEEtrantools}
\usepackage{enumerate}
\author{Daniel Xu}
\title{MATH 313 HMWK 11}
\begin{document}
\maketitle
\textbf{Exercise 5.3.8}
We want to show that
\begin{equation*}
  f'(0) = \lim_{x \rightarrow 0} \frac{f(x) - f(0)}{x - 0}
\end{equation*}
exists. Let us define functions \(m(x) = f(x) - f(0)\) and \(n(x) = x\) with \(m\) and \(n\)
continuous on the same interval that \(f\) is. Then \(m(0) = f(0) - f(0) = 0\) and
\(n(0) = 0\). It is evident that
\begin{equation*}
  f'(0) = \lim_{x \rightarrow 0} \frac{m(x)}{n(x)}
\end{equation*}
and
\begin{IEEEeqnarray*}{r C l"s}
  \lim_{x \rightarrow 0} \frac{m'(x)}{n'(x)} & = & \lim_{x \rightarrow 0} \frac{f'(0)}{1} \\
  & = & \lim_{x \rightarrow 0} f'(0) \\
  & = & L & this is given
\end{IEEEeqnarray*}

By L'Hospital's Rule,
\begin{equation*}
  \lim_{x \rightarrow 0} \frac{m(x)}{n(x)}  = L
\end{equation*}

and since
\begin{equation*}
  f'(0) = \lim_{x \rightarrow 0} \frac{m(x)}{n(x)} \\
\end{equation*}

\(f'(0) = L\). 

\textbf{Exercise 5.3.11}
\begin{enumerate}[(a)]
\item Say that a sequence \(x_{n} \rightarrow a\), with \(x_{n} > 0\). Then
  \begin{IEEEeqnarray*}{r C l"s}
    \frac{f \left( x_{n} \right)}{g \left( x_{n} \right)} & = & \frac{f \left( x_{n} \right) - 0}{g \left( x_{n} \right) - 0} \\
    & = & \frac{f \left( x_{n} \right) - f(a)}{g \left( x_{n} \right) - g(a)} \\
    & = & \frac{f' \left(c_{n} \right)}{g' \left( c_{n} \right)} \\
    & \rightarrow & L & by definition of limit
  \end{IEEEeqnarray*}
\item No, it does not necessarily follow. Let
  \begin{equation*}
    f(x) = \frac{1}{6} x^{5} \quad \textrm{and} \quad g(x) = \frac{1}{2} x^{2} - 3x
  \end{equation*}
  Then
  \begin{IEEEeqnarray*}{r C l}
    \lim_{x \rightarrow 3} \frac{f(x)}{g(x)} & = & \lim_{x \rightarrow 3} \frac{\dfrac{1}{6} x^{5}}{\dfrac{1}{2} x^{2} - 3x} \\
    & = & \frac{\dfrac{243}{6}}{-\dfrac{9}{2}} \\
    & = & \frac{243}{6} \cdot - \frac{2}{9} \\
    & = & -9
  \end{IEEEeqnarray*}
  However,
  \begin{IEEEeqnarray*}{r C l}
    \lim_{x \rightarrow 3} \frac{f'(x)}{g'(x)} & = & \lim_{x \rightarrow 3} \frac{x^{5}}{x - 3} \\
    & = & \infty
  \end{IEEEeqnarray*}
  
\end{enumerate}

\textbf{Exercise 6.2.1}
\begin{enumerate}[(a)]
\item
  \begin{IEEEeqnarray*}{r C l}
    \lim_{n \rightarrow \infty} g_{n} (x) & = & \lim_{n \rightarrow \infty} \frac{nx}{1 + nx^{2}} \\
    & = & \lim_{n \rightarrow \infty} \frac{nx}{n \cdot \left(\dfrac{1}{n} + x^{2} \right)} \\
    & = & \lim_{n \rightarrow \infty} \frac{x}{\dfrac{1}{n} + x^{2}} \\
    & = & \frac{1}{x}
  \end{IEEEeqnarray*}
\item Yes. For any \(c \in (0, 1)\), given any arbitrary \(\epsilon > 0\), we can just choose
  \(\delta = |c| \cdot \epsilon \). Then, since
  \begin{equation*}
    \left| \frac{x - c}{x} \right| < \left| x - c \right|
  \end{equation*}
  then
  \begin{equation*}
    \left|\frac{ c - x }{cx} \right| < \epsilon
    \end{equation*}
\item Yes. See above. 
\item Yes. See above.
\end{enumerate}

  

\end{document}
