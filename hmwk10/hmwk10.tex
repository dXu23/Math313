\documentclass{article}
\usepackage[retainorgcmds]{IEEEtrantools}
\usepackage{amsmath}
\usepackage{amssymb}
\usepackage{enumerate}
\author{Daniel Xu}
\title{MATH 313 HMWK 10}
\begin{document}
\maketitle
\textbf{Exercise 5.2.3}
\begin{enumerate}[(a)]
\item
  \begin{IEEEeqnarray*}{r C l}
    h'(c) & = & \lim_{x \rightarrow c} \frac{h(c) - h(x)}{c - x} \\
    & = & \lim_{x \rightarrow c} \frac{\dfrac{1}{c} - \dfrac{1}{x}}{c - x} \\
    & = & \lim_{x \rightarrow c} \frac{\dfrac{x - c}{cx}}{c - x} \\
    & = & \lim_{x \rightarrow c} - \frac{c - x}{cx} \cdot \frac{1}{c - x} \\
    & = & \lim_{x \rightarrow c} \frac{1}{cx} \\
    & = & \frac{1}{c^{2}}
  \end{IEEEeqnarray*}
\item
  \begin{IEEEeqnarray*}{r C l}
    (f / g)' (c) & = & \left(f \cdot \left( 1 / g \right) \right) (c) \\
    & = & f'(c) \cdot \frac{1}{g(c)} + f(c) \cdot \left( 1 / g \right)' (c) \\
    & = & \frac{f'(c)}{g(c)} - \frac{f(c) \cdot g'(c)}{[g(c)]^{2}} \\
    & = & \frac{f'(c) \cdot g(c)}{[g(c)]^{2}} - \frac{f(c) \cdot g'(c)}{[g(c)]^{2}} \\
    & = & \frac{f'(c) \cdot g(c) - f(c) \cdot g'(c) }{[g(c)]^{2}}
  \end{IEEEeqnarray*}
  

\item
  \begin{IEEEeqnarray*}{r C l}
    (f/g)'(c) & = & \lim_{x \rightarrow c} \frac{(f / g) (x) - (f / g) (c)}{x - c} \\
    & = & \lim_{x \rightarrow c} \frac{\dfrac{f(x)}{g(x)} - \dfrac{f(c)}{g(c)}}{x - c} \\
    & = & \lim_{x \rightarrow c} \frac{\dfrac{f(x) \cdot g(c)}{g(x) \cdot g(c)} - \dfrac{f(c) \cdot g(x)}{g(c) \cdot g(x)}}{x - c} \\
    & = & \lim_{x \rightarrow c} \frac{1}{x - c} \cdot \frac{1}{g(x) \cdot g(c)} \cdot \left(f(x) \cdot g(c) - f(c) \cdot g(x) \right) \\
    & = & \lim_{x \rightarrow c} \frac{1}{x - c} \cdot \frac{1}{g(x) \cdot g(c)} \cdot \left(f(x) \cdot g(c) + f(c) \cdot g(c) - f(c) \cdot g(c) - f(c) \cdot g(x) \right) \\
    & = & \lim_{x \rightarrow c} \frac{1}{g(x) \cdot g(c)}
    \cdot \left(\frac{f(x) \cdot g(c) - f(c) \cdot g(c)}{x - c} - \frac{g(x) \cdot f(c) - g(c) \cdot f(c)}{x - c}\right) \\
    & = & \lim_{x \rightarrow c} \frac{g(c) \cdot f'(c) - f(c) \cdot g'(c)}{[g(c)]^{2}}
  \end{IEEEeqnarray*}
\end{enumerate}

\textbf{Exercise 5.2.7}
\begin{enumerate}[(a)]
\item
  \begin{IEEEeqnarray*}{r C l}
    g'_{a} (0) & = & \lim_{x \rightarrow 0} \frac{g_{a} (x) - g_{a} (0)}{ x - 0 } \\
    & = & \lim_{x \rightarrow 0} \frac{x^{a} \sin(1 / x) - 0}{ x - 0 } \\
    & = & \lim_{x \rightarrow 0} x^{a -1} \sin(1 / x) 
  \end{IEEEeqnarray*}
  In this situation, if \(a = 0\) or \(a = 1\), the limit will not exist. We
  choose \(a = 2\). 
\end{enumerate}

\textbf{Exercise 5.3.2}
We take the contrapositve of Rolle's theorem. If for all points \(c\) in our
interval \(A\) \(f'(c) \neq 0\), then for any two distinct points \(a\) and \(b\) in
our interval, \(f(a) \neq f(b)\). Since \(a \neq b \implies f(a) \neq f(b)\),
then \(f\) is one-to-one on \(A\). The function \(f(x) = x^{3}\) is one-to-one
on the interval \([-1, 1]\), yet \(f'(0) = 0\).

\textbf{Exercise 5.3.3}

\textbf{Exercise 5.3.4}
\begin{enumerate}[(a)]
\item Let \(f \left( x_{n} \right) \rightarrow L\). Then by the definition of the limit
  for a sequence, \(\forall \epsilon > 0 \exists \delta \forall n \geq N \quad \left|f \left( x_{n} - L \right) \right| < \epsilon\).
  Since \(f \left( x_{n} \right) = 0\) for all \(n \geq N\), then we are left with
  \(\left| -L \right| < \epsilon\). The only way that \(\left| -L \right| < \epsilon\)
  is that \(L = 0\). So \(f \left(x_{n} \right) \rightarrow 0 = f(0)\) by the characterizations
  of continuity (\(f\) must be continuous on the interval since it is differentiable).

  When we evaluate \(f'(0)\), we get
  \begin{IEEEeqnarray*}{r C l"s}
    f'(0) & = & \lim_{x \rightarrow 0} \frac{f(x) - f(0)}{x - 0} \\
    & = & \lim_{x \rightarrow 0} \frac{f(x)}{x} & since \(f(0) = 0\)
  \end{IEEEeqnarray*}
  Now we must show that \(\lim_{x \rightarrow 0} f(x) / x = 0\). We can do this by showing
  that for all \(\epsilon > 0\), there exists \(\delta\) such that
  \(\left| x - 0 \right| < \delta \implies \left| f(x) / x - 0 \right| < \epsilon\).

\end{enumerate}

  

\end{document}
