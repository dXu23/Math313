\documentclass{article}
\usepackage{amsmath}
\usepackage{amssymb}
\usepackage{enumerate}
\usepackage[retainorgcmds]{IEEEtrantools}
\author{Daniel Xu}
\title{MATH 313 HMWK 8}
\begin{document}
\maketitle
\textbf{Exercise 4.2.2}
\begin{enumerate}[(a)]
\item
  \begin{IEEEeqnarray*}{r C l}
    \left| f(x) - 9 \right| & < & 1 \\
    \left| 5x - 6 - 9 \right| & < & 1 \\
    \left| 5x - 15 \right| & < & 1 \\
    5 \cdot \left| x - 3 \right| & < & 1 \\
    \left| x - 3 \right| & < & \frac{1}{5} \\
  \end{IEEEeqnarray*}
  The largest possible \(\delta\)-neighborhood is \(V_{1/5} (3)\). 
\item
  \begin{IEEEeqnarray*}{r C l}
    \left| f(x) - 2 \right| & < & 1 \\
    \left| \sqrt{x} - 2 \right| & < & 1 \\
    -1 < \sqrt{x} - 2 & < & 1 \\
    1 < \sqrt{x} & < & 3 \\
    1 < x & < & 9 \\
    -4 < x - 5 & < & 4 \\
    \left| x - 5 \right| & < & 4
  \end{IEEEeqnarray*}
  The largest \(\delta\)-neighborhood is \(V_{4} (5)\).

\item If \(x\) is less than 3, then \([[x]]\) will be 2, which is
  1 away from 3, so it will be out of the \(\epsilon\)-neighborhood. Thus,
  our \(x\) must lie in a \(\delta\)-neighborhood whose infimum is greater than 3.
  Thus, the largest \(\delta\)-neighborhood is \(V_{\pi - 3} (\pi)\).

\item If \(x\) is less than 3, then \([[x]]\) will be 2, which is
  more than 0.01 away from 3, so it will be out of the \(\epsilon\)-neighborhood. Thus,
  our \(x\) must lie in a \(\delta\)-neighborhood whose infimum is greater than 3.
  Thus, the largest \(\delta\)-neighborhood is \(V_{\pi - 3} (\pi)\), so it is the same
  as the last problem. 
\end{enumerate}
\textbf{Exercise 4.2.3}
\begin{enumerate}[(a)]
\item Let \(x_{n} = 1 - 1/n\), \(y_{n} = 1 + 1 / n\), and \(z_{n} = 1 - (1 / 2)^{n}\).
\item
  \begin{IEEEeqnarray*}{r C l}
    x_{n} & = & 1 - \frac{1}{n} \\
    & = & \frac{n}{n} - \frac{1}{n} \\
    & = & \frac{n - 1}{n}
  \end{IEEEeqnarray*}
  By Thomae's function, \(t\left (x_{n} \right) = 1 / n\), since \(n - 1\) and \(n\)
  are relatively prime. 
  \begin{IEEEeqnarray*}{r C l}
    y_{n} & = & 1 + \frac{1}{n} \\
    & = & \frac{n}{n} + \frac{1}{n} \\
    & = & \frac{n + 1}{n}
  \end{IEEEeqnarray*}
  By Thomae's function, \(t \left(y_{n} \right) = 1 / n\), since \(n + 1\) and \(n\)
  are relatively prime. 

  \begin{IEEEeqnarray*}{r C l}
    z_{n} & = & 1 - \frac{1}{2^{n}} \\
    & = & \frac{2^{n}}{2^{n}} - \frac{1}{2^{n}} \\
    & = & \frac{2^{n} - 1}{2^{n}}
  \end{IEEEeqnarray*}
  By Thomae's function, \(t \left(z_{n} \right) = 1 / 2^{n}\), since \(2^{n} - 1\) and \(2^{n}\)
  are relatively prime. It should be evident that \(\lim t \left(x_{n} \right) = 0\),
  \(\lim t \left(y_{n} \right) = 0\), and \(\lim t \left(x_{n} \right) = 0\).
\item As evidenced by the last problem, it seems that \(\lim_{t \rightarrow 1} t(x) = 0\).
  We consider the set \(\{ x \in \mathbb{R} : t(x) \geq \epsilon \}\). Such a set
  would contain rational numbers in the form of \(m / n\), where \(1 / n \geq \epsilon\),
  with the exception of \(x = 0\) and \(\epsilon < 1\). The set contains isolated points,
  which are points that are not limit points of the set. Therefore, the set is not closed. 
  
\end{enumerate}
\textbf{Exercise 4.2.6}
\begin{enumerate}[(a)]
\item True. Making \(\delta\) less means that values of \(f(x)\) will lie in
  a smaller interval than before, and the smaller interval will still be
  in the \(\epsilon\)-neighborhood. 
\item False. Consider a function \(f : \mathbb{R} \rightarrow \mathbb{R}\):
  \[f(x) =
  \begin{cases}
    x & x \neq 0 \\
    2 & x = 0
  \end{cases}
  \]
  It should be evident that \(\lim_{x \rightarrow 0} f(x) = 0\), but \(f(x) = 2\), even though
  \(2 \in \mathbb{R}\). 
\item True.
  \begin{IEEEeqnarray*}{r C l"s}
    \lim_{x \rightarrow a} 3 [f(x) - 2]^{2} & = & 3 \lim_{x \rightarrow a} [f(x) - 2]^{2} & by (i) of Corollary 4.2.4 \\
    & = & 3 \lim_{x \rightarrow a} [f(x) - 2] [f(x) - 2] \\
    & = & 3 \cdot \left(\lim_{x \rightarrow a} f(x) - 2 \right) \left(\lim_{x \rightarrow a} f(x) - 2 \right) & by (iii) of Corollary 4.2.4 \\
    & = & 3 \cdot (L - 2)^{2}
    \end{IEEEeqnarray*}
  \item False. \(\lim_{x \rightarrow \infty} 1 / x = 0\), but \(\lim_{x \rightarrow \infty} e^{x} / x \neq 0\). 
  \end{enumerate}
\textbf{Exercise 4.2.11}
Say we are given some \(\epsilon\). By the definition of the limit, we know that
\begin{IEEEeqnarray*}{r C l}
  \forall \epsilon \geq 0 \quad \exists \delta_{1} : 0 < \left| x - c \right| < \delta_{1} \implies \left| f(x) - L \right| < \epsilon \\
  \forall \epsilon \geq 0 \quad \exists \delta_{2} : 0 < \left| x - c \right| < \delta_{2} \implies \left| h(x) - L \right| < \epsilon
\end{IEEEeqnarray*}

We are given that \(f(x) \leq g(x) \leq h(x)\). Then
\begin{IEEEeqnarray*}{r C l}
  - \epsilon & < & f(x) - L < \epsilon \\
  - \epsilon & < & g(x) - L < \epsilon \\
  f(x) & \leq & g(x) \leq h(x) \\
  f(x) - L & \leq & g(x) - L \leq h(x) - L \\
  - \epsilon < f(x) - L & \leq & g(x) - L \leq h(x) - L < \epsilon \\
  - \epsilon & < & g(x) - L < \epsilon \\
  \left| g(x) - L \right| & < & \epsilon
\end{IEEEeqnarray*}
And for our \(\delta\) in the case of \(g(x)\), we choose \(\delta = \min\{\delta_{1}, \delta_{2}\}\)

\end{document}
