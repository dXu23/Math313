\documentclass{article}
\usepackage{amsmath}
\usepackage{amssymb}
\usepackage[retainorgcmds]{IEEEtrantools}
\usepackage{enumerate}
\begin{document}
\textbf{Problem 2.3.4}
\begin{enumerate}[(a)]
\item
  \begin{IEEEeqnarray*}{r C l}
    \lim \left(\frac{1 + 2a_{n}}{1 + 3a_{n} - 4a_{n}^{2}}\right) & = & \frac{\lim (1 + 2a_{n})}{\lim (1 + 3a_{n} - 4a_{n}^{2})} \\
    & = & \frac{1}{1} \\
    & = & 1
  \end{IEEEeqnarray*}
\item
  \begin{IEEEeqnarray*}{r C l}
    \lim \left( \frac{(a_{n} + 2)^{2} - 4}{a_{n}} \right) & = & \lim \left( \frac{a_{n}^{2} + 4 a_{n} + 4 - 4}{a_{n}} \right) \\
    & = & \lim \left( \frac{a_{n}^{2} + 4 a_{n}}{a_{n}}\right) \\
    & = & \lim (a_{n} + 4) \\
    & = & \lim (a_{n}) + 4 \\
    & = & 0 + 4 \\
    & = & 4
  \end{IEEEeqnarray*}
\item
  \begin{IEEEeqnarray*}{r C l}
    \lim \left(\frac{\frac{2}{a_{n}} + 3}{\frac{1}{a_{n}} + 5} \right) & = & \lim \left(\frac{\frac{2}{a_{n}} + 3}{\frac{1}{a_{n}} + 5} \cdot \frac{a_{n}}{a_{n}} \right) \\
    & = & \lim \left(\frac{2 + 3 a_{n}}{1 + 5 a_{n}} \right) \\
    & = & \left(\frac{ \lim (2 + 3 a_{n})}{ \lim (1 + 5 a_{n}} \right) \\
    & = & 2
    \end{IEEEeqnarray*}
\end{enumerate}

\textbf{Problem 2.3.9}
\begin{enumerate}[(a)]

\item If \((a_{n})\) is bounded, then there exists a number \(M > 0\) such
  that \(|a_{n}| \leq M\). If \(\lim b_{n} = 0\), then for all \(\epsilon_{0} > 0\),
  there exists a \(N_{0} \in \mathbb{N}\) such that whenever \(n \geq N_{0}\) it
  follows that \(|b_{n} - 0| < \epsilon_{0}\). Then
  \begin{IEEEeqnarray*}{r C l}
    |b_{n} - 0| & < & \epsilon_{0} \\
    |b_{n}| & < & \epsilon_{0} \\
    |a_{n}| \cdot |b_{n}| & < & M \cdot \epsilon_{0} \\
    |a_{n} \cdot b_{n} | & < & M \cdot \epsilon_{0} \\
    |a_{n} \cdot b_{n} - 0 | & < & M \cdot \epsilon_{0}
  \end{IEEEeqnarray*}
  Whatever \(\epsilon\) is chosen, we can find a \(N\) for \(\epsilon\) for the
  sequence \((a_{n} b_{n})\) satisfying the definition of convergence in the following
  way. Divide the \(\epsilon\) by the upper bound \(M\) to get a \(\epsilon_{0}\) and
  find a \(N_{0}\) for that \(\epsilon\) for the case of \(b_{n}\). Then, we can use
  the same \(N_{0}\) as \(N\) for \(a_{n} b_{n}\). Thus, \(\lim (a_{n} b_{n}) = 0\) since
  a \(N\) can be found for any \(\epsilon\) that satisfies the condition of convergence.
  We are not allowed to use the Algebraic Limit Theorem because it is not known whether
  \(a_{n}\) is convergent.
\end{enumerate}

\textbf{Problem 2.3.11}

\textbf{Problem 2.4.3}
We first show that the limit of the sequence exists using the monotone convergence theorem.
We proceed by induction to show that the sequence is monotone; that is, \(\forall n x_{n + 1} > x_{n}\).
We consider the base case.
\begin{IEEEeqnarray*}{r C l}
  \sqrt{2 + \sqrt{2}} & \stackrel{?}{\geq} & \sqrt{2} \\
  2 + \sqrt{2} & > & 2
\end{IEEEeqnarray*}

We now proceed to the inductive step. We must show that if \(x_{k+1} \geq x_{k}\) for
some particular \(k\), then it will also be true for \(x_{k + 2} \geq x_{k + 1}\).
So
\begin{IEEEeqnarray*}{r C l}
  x_{k + 1} & > & x_{k} \\
  2 + x_{k + 1} & > & 2 + x_{k} \\
  \sqrt{2 + x_{k + 1}} & > & \sqrt{2 + x_{k}} \\
  x_{k + 1} & > & x_{k}
\end{IEEEeqnarray*}

Now that we have show that \((x_{n}\) is monotone, we must now show that
it is bounded. We show that \(x_{k} < 16\). We first consider the base case.
\(x_{1} = \sqrt{2}\) is indeed less than 16. We can now proceed to the
inductive step, where we show that if \(x_{k} < 16\), then \(x_{k + 1} < 16\).
We first assume to the contrary that \(x_{k + 1} \geq 16\). Then
\begin{IEEEeqnarray*}{r C l}
  x_{k + 1} & \geq & 16 \\
  \sqrt{2 + x_{k}} & \geq & 16 \\
  2 + x_{k} & \geq & 256 \\
  x_{k} & \geq 254
\end{IEEEeqnarray*}
However, this contradicts our original assumption that \(x_{k} < 16\). Thus,
we deduce that \(x_{n}\) converges to some limit \(L\). Now we define a sequence
\(y_{n} = x_{n + 1}\). \(y_{n}\), which is a subsequence of \(x_{n}\), converges to
the same limit that \(y_{n}\) does. So
\begin{IEEEeqnarray*}{r C l}
  x_{n} & \rightarrow & y \\
  2 + x_{n} & \rightarrow & 2 + y \\
  \sqrt{5 + x_{n}} & \rightarrow & \sqrt{2 + y} \\
  y_{n} & \rightarrow & \sqrt{2 + y} \\
  \end{IEEEeqnarray*}
Then
  
  \begin{IEEEeqnarray*}{r C l}
    y & = & \sqrt{2 + y} \\
    y^{2} & = & 2 + y \\
    y^{2} - y - 2 & = & 0 \\
    (y + 1)(y - 2) & = & 0 \\
    y & = & 2 \\
  \end{IEEEeqnarray*}

  \textbf{Problem 2.4.5}
  \begin{enumerate}[(a)]
  \item We proceed by induction. For the base case, \(x_{1}^{2} = 2^{2} = 4 \geq 2\). Now
    for the inductive step. We must now show that if \(x_{k}^{2} \geq 2\), then
    \(x_{k+1}^{2} \geq 2\). So
    \begin{IEEEeqnarray*}{r C l}
      x_{k}^{2} & \geq & 2 \\
      x_{k}^{2} - 2 & \geq & 0 \\
      \left(x_{k}^{2} - 2 \right)^{2} & \geq & 0 \\
      x_{k}^{4} - 4 x_{k}^{2} + 4 & \geq & 0 \\
      x_{k}^{2} - 4 + \frac{4}{x_{k}^{2}} & \geq & 0 \\
      x_{k}^{2} + 4 + \frac{4}{x_{k}^{2}} & \geq & 8 \\
      \frac{1}{4} \left(x_{k}^{2} + 4 + \frac{4}{x_{k}^{2}} \right) & \geq & 2 \\
      \frac{1}{4} \left(x_{k} + \frac{2}{x_{k}} \right)^{2} & \geq & 2 \\
      x_{k + 1}^{2} & \geq & 2
    \end{IEEEeqnarray*}
    Thus, we know that \(x_{n}^{2} \geq 2\). Now, we have to show that \(x_{n} - x_{n + 1} \geq 0\).
    So
    \begin{IEEEeqnarray*}{r C l}
      x_{n}^{2} & \geq & 2 \\
      x_{n}^{2} - 2 & \geq & 0 \\
      \frac{1}{2 x_{n}} \cdot \left(x_{n}^{2} - 2 \right) & \geq & \frac{1}{2 x_{n}} \cdot 0 \\
      \frac{1}{2} x_{n} - \frac{1}{x_{n}} & \geq 0 \\
      \frac{1}{2} x_{n} + \frac{1}{2} x_{n} - \frac{1}{2} x_{n} - \frac{1}{x_{n}} & \geq & 0 \\
      x_{n} - \frac{1}{2} \left(x_{n} + \frac{2}{x_{n}} \right) & \geq & 0 \\
      x_{n} - x_{n + 1} & \geq & 0 
    \end{IEEEeqnarray*}
  \item Let \(x_{1} = c\) and define
    \[x_{n + 1} = \frac{x_{n} + \frac{x_{n}}{c}}{2}\]
  \end{enumerate}
  \textbf{Problem 2.4.8}
  \begin{enumerate}[(a)]
  \item A explicit formula would be \(1 - \frac{1}{2^{n}}\). The sequence converges to 1.
  \item A explicit formula would be \(\frac{n}{n + 1}\). The sequence converges to 1. 
    \item A explicit formula would be \(\log (n + 1)\). 
  \end{enumerate}

  \textbf{Problem 2.5.1}
  \begin{enumerate}[(a)]
  \item Impossible. According to the Bolzano-Weierstrass Theorem, every bounded
    sequence contains a convergent subsequence. Since the sequence has a bounded
    subsequence, the bounded subsequence contains a subsubsequence that converges
    which is itself a subsequence of the original sequence.

  \item Define
    \[x_{n} = \frac{1}{2} + \frac{1}{2} (-1)^{k} \frac{k}{k + 1}\]
  \item
  \item Impossible. Such a sequence would contain a subsequence that converges to
    zero, since the sequence \(x_{n} = 1 / n\) itself converges to zero, which is
    a point outside of the set \(\{1, 1/2, 1/3, 1/4, 1/5, \ldots\}\). 
  \end{enumerate}
  
  \textbf{Exercise 2.5.2}
  \begin{enumerate}[(a)]
  \item True. Define a proper subsequence \((y_{n})\) of \((x_{n})\) such that
    \(y_{n} = x_{n + 1}\).
  \item True. Consider the contrapositive of the statement, if \((x_{n})\)
    converges, then \(x_{n}\) contains a convergent subsequence. Since \(x_{n}\)
    is a subsequence of itself, then \(x_{n}\) will always have a convergent
    subsequence. Since the contrapositive of the statement is true, the statement
    is true.
  \item True.
    \item False. 
   
  \end{enumerate}
  

\end{document}
