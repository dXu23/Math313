\documentclass{article}
\usepackage{amsmath}
\usepackage[retainorgcmds]{IEEEtrantools}
\usepackage{amssymb}
\author{Daniel Xu}
\title{MATH 313 HMWK 3}
\begin{document}
\maketitle
\textbf{Exercise 2.2.1} An example of a vercongent sequence would be
\(x_{n} = \sin x\). Let \(\epsilon = \frac{101}{100}\) and let \(x = 0\).
Then \(x_{n}\) converges to 0, since \(\left| \sin x - 0 \right| < \frac{101}{100} \).
Since \(x_{n}\) does not converge to any number, it is a vercongent sequence
that is also divergent. What a vercongent sequence describes is a sequence
for whom all values lie in a certain bound, namely \((x - \epsilon, x + \epsilon)\).

\textbf{Excercise 2.2.2}
\begin{enumerate}
\item Let \(\epsilon > 0\) be arbitrary. Now choose \(N\) such that
  \[N > \frac{3}{25 \epsilon} - \frac{4}{5}\]
  To verify that our choice of \(N\) is appropiate, let \(n \in \mathbb{N}\)
  satisfy \(n \geq N\). Then, \(n \geq N\) implies that
  \begin{IEEEeqnarray*}{r C l}
    n & > & \frac{3}{25 \epsilon} - \frac{4}{5} \\
    5n & > & \frac{3}{5 \epsilon} - 4 \\
    5n + 4 & > & \frac{3}{5 \epsilon} \\
    \epsilon & > & \frac{3}{5 \cdot (5n + 4)} \\
    \epsilon & > & \frac{2 \cdot (5n + 4) - 5 \cdot (2n + 1)}{5 \cdot (5n + 4)} \\
    \epsilon & > & \frac{2 \cdot (5n + 4)}{5 \cdot (5n + 4)} - \frac{5 \cdot (2n + 1)}{5 \cdot (5n + 4)} \\
    \epsilon & > & \frac{2}{5} - \frac{2n +1}{5n + 4} \\
    \epsilon & > & \left| \frac{2}{5} - \frac{2n + 1}{5n + 4} \right| \\
    \left| \frac{2}{5} - \frac{2n + 1}{5n + 4} \right| & < & \epsilon \\
    \left| \frac{2n + 1}{5n + 4} - \frac{2}{5} \right| & < & \epsilon
  \end{IEEEeqnarray*}

\item

  
\end{enumerate}
\textbf{2.3.1}
\begin{enumerate}
\item If \(x_{n} \rightarrow 0\), then there exists \(\epsilon_{0} > 0\) such that
  there exists \(N \in \mathbb{N}\) for all \(n \geq \mathbb{N}\),
  \(x_{n} - 0 < \epsilon_{0}\). 
\end{enumerate}

\textbf{2.3.1}
\begin{enumerate}
\item If \(x_{n} \rightarrow 0\), then it must be true that for all \(\epsilon_{0}\)
  , there exists a \(N \in \mathbb{N}\) such that for all \(n \geq N\),
  \(\left| x_{n} - 0 \right| < \epsilon_{0}\). Then
  \begin{IEEEeqnarray*}{r C l"s}
    \left| x_{n} - 0 \right| & < & \epsilon_{0} \\
    x_{n} & < & \epsilon_{0} & since \(x_{n} \geq 0\) \\
    \sqrt{x_{n}} & < & \sqrt{\epsilon_{0}}
    \end{IEEEeqnarray*}
\end{enumerate}

\textbf{2.3.5} If \(x_{n}\) and \(y_{n}\) are both convergent, then there
must exist a \(N_{1} , N_{2} \in \mathbb{N}\) such that for all \(n_{1} \geq N_{2}\)
and for all \(n_{2} \geq N_{2}\), \(\left| a_{n} - L \right| < \epsilon_{1}\) and
\(\left| b_{n} - L \right| < \epsilon_{2}\) for any \(\epsilon > 0\) since \(a_{n}\)
and \(b_{n}\) both converge to the same limit. In the case of \(z_{n}\), we choose
\(N = \max(N_{1}, N_{2})\). Then for all \(n \geq N\), we know that
\(\left|z_{n} - L \right| < \epsilon\).

Now we must prove the converse. We start with the fact that \(z_{n}\) is convergent.
Then by definition, for any \(\epsilon > 0\) 


\textbf{Exercise 2.3.7}
\begin{enumerate}
\item Let \(x_{n} = n\) and \(y_{n} = -n\). Both \(x_{n}\) and \(y_{n}\) diverge. However,
  \(x_{n} + y_{n}\) converge to zero.

\item Impossible by Algebraic Limit Theorem ii.
  
\item Let \(b_{n} = \frac{1}{n}\) with \(b_{n} \neq 0\) for all \(n \in \mathbb{N}\). However,
  \((1 / b_{n})\) diverges.
  
\item Impossible. By Theorem 2.3.2, \(a_{n}\) must be a divergent sequence, so then \((a_{n} - b_{n}\)
  is a sequence that diverges.
  
\item Let \(a_{n} = \frac{1}{n}\) and \(b_{n} = n\). \(a_{n}\) converges to 0 and \(b_{n}\)
  diverges. \((a_{n} b_{n})\) converges to 1.

\end{enumerate}



\end{document}
