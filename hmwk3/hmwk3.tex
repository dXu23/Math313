\documentclass{article}
\usepackage{amsmath}
\usepackage[retainorgcmds]{IEEEtrantools}
\usepackage{amssymb}
\usepackage{enumerate}
\author{Daniel Xu}
\title{MATH 313 HMWK 3}
\begin{document}
\maketitle
\textbf{Exercise 2.2.1} An example of a vercongent sequence would be
\(x_{n} = \sin x\). Let \(\epsilon = \frac{101}{100}\) and let \(x = 0\).
Then \(x_{n}\) converges to 0, since \(\left| \sin x - 0 \right| < \frac{101}{100} \).
Since \(x_{n}\) does not converge to any number, it is a vercongent sequence
that is also divergent. What a vercongent sequence describes is a sequence
for whom all values lie in a certain bound, namely \((x - \epsilon, x + \epsilon)\).

\textbf{Excercise 2.2.2}
\begin{enumerate}[(a)]
\item Let \(\epsilon > 0\) be arbitrary. Now choose \(N\) such that
  \[N > \frac{3}{25 \epsilon} - \frac{4}{5}\]
  To verify that our choice of \(N\) is appropiate, let \(n \in \mathbb{N}\)
  satisfy \(n \geq N\). Then, \(n \geq N\) implies that
  \begin{IEEEeqnarray*}{r C l}
    n & > & \frac{3}{25 \epsilon} - \frac{4}{5} \\
    5n & > & \frac{3}{5 \epsilon} - 4 \\
    5n + 4 & > & \frac{3}{5 \epsilon} \\
    \epsilon & > & \frac{3}{5 \cdot (5n + 4)} \\
    \epsilon & > & \frac{2 \cdot (5n + 4) - 5 \cdot (2n + 1)}{5 \cdot (5n + 4)} \\
    \epsilon & > & \frac{2 \cdot (5n + 4)}{5 \cdot (5n + 4)} - \frac{5 \cdot (2n + 1)}{5 \cdot (5n + 4)} \\
    \epsilon & > & \frac{2}{5} - \frac{2n +1}{5n + 4} \\
    \epsilon & > & \left| \frac{2}{5} - \frac{2n + 1}{5n + 4} \right| \\
    \left| \frac{2}{5} - \frac{2n + 1}{5n + 4} \right| & < & \epsilon \\
    \left| \frac{2n + 1}{5n + 4} - \frac{2}{5} \right| & < & \epsilon
  \end{IEEEeqnarray*}

\item Let \(\epsilon > 0\) be arbitrary and let \(a_{n} = 2n^{2} / \left(n^{3} + 3\right)\).
  Choose \(N \in \mathbb{N}\) with \(N > \frac{4}{\epsilon^{2}}\). Now, we must show that for
  all \(n \in \mathbb{N}\), \(n \geq N\) satisfies \(\left|a_{n} - L \right| < \epsilon\).
  If \(n \geq N\), then
  \begin{IEEEeqnarray*}{r C l}
    n & > & \frac{4}{\epsilon^{2}} \\
    \sqrt{n} & > & \frac{2}{\epsilon} \\
    \frac{2}{\sqrt{n}} & < \epsilon
  \end{IEEEeqnarray*}
  If we can show that \(2 / \sqrt{n} > \frac{2n^{2}}{n^{3} + 3}\), then we
  are done. So,
  \begin{IEEEeqnarray*}{r C l}
    \frac{2n^{2}}{n^{3} + 3} & < & \frac{2}{\sqrt{n}} \\
    \frac{1}{n^{3} + 3} & < & \frac{1}{\sqrt{n^{5}}} \\
    n^{3} + 3 & > & \sqrt{n^{5}}
  \end{IEEEeqnarray*}
  The last line is a true statement, so we are done. 
  
  
\item Let \(\epsilon > 0\) be arbitrary. Choose \(N \in \mathbb{N}\) with
  \(N > \frac{1}{\epsilon^{3}} \). To verify that the choice of \(N\) was
  appropiate, let \(n \in \mathbb{N}\) satisfy \(n \geq N\). Then
  \begin{IEEEeqnarray*}{r C l}
    n & \geq & \frac{1}{\epsilon^{3}} \\
    \frac{1}{n} & \leq & \epsilon^{3} \\
    \sqrt[3]{\frac{1}{n}} & \leq & \sqrt[3]{\epsilon^{3}} \\
    \frac{1}{\sqrt[3]{n}} & \leq & \epsilon \\
    \frac{\sin \left(n^{2} \right)}{\sqrt[3]{n}} & \leq & \sin \left(n^{2} \right) \cdot \epsilon
  \end{IEEEeqnarray*}
  However, \(-1 \leq \sin x \leq 1\), so
  \[\left| \frac{\sin \left(n^{2} \right)}{\sqrt[3]{n}} - 0 \right| \leq \epsilon\]

  
\end{enumerate}
\textbf{Exercise 2.3.1}
\begin{enumerate}[(a)]
\item If \(x_{n} \rightarrow 0\), then for any \(\epsilon_{0} > 0\) 
  there exists \(N_{0} \in \mathbb{N}\) such that for all \(n \geq \mathbb{N_{0}}\),
  \(|x_{n} - 0| < \epsilon_{0}\). Since \(x_{n} \geq 0\), then
  \begin{IEEEeqnarray*}{r C l}
    x_{n} & < & \epsilon_{0} \\
    \sqrt{x_{n}} & < & \sqrt{\epsilon_{0}} \\
    \sqrt{x_{n}} - 0 & < & \sqrt{\epsilon_{0}} \\
    \left|\sqrt{x_{n}} - 0 \right| & < & \sqrt{\epsilon_{0}} \\
  \end{IEEEeqnarray*}
  Now, we can just set any \(\epsilon\) given to us to \(\sqrt{\epsilon_{0}}\).
  To supply a given \(N\) for an \(\epsilon\), we can use the \(N_{0}\) supplied
  to us by \(\epsilon_{0}\).

\item If \(x_{n} \rightarrow x\), then for any \(\epsilon_{0} > 0\), there exists
  \(N_{0} \in \mathbb{N_{0}}\) such that for all \(n \geq \mathbb{N}\),
  \(\left| x_{n} - x \right| < \epsilon_{0}\). Since \(x_{n} \geq 0\), so is \(x\).
  Then
  \begin{IEEEeqnarray*}{r C l}
    x_{n} - x & < & \epsilon_{0} \\
    x_{n} & < & \epsilon_{0} + x \\
    \sqrt{x_{n}} & < & \sqrt{\epsilon_{0} + x} \\
    \sqrt{x_{n}} - \sqrt{x} & < & \sqrt{\epsilon_{0} + x} - \sqrt{x} \\
    \left|\sqrt{x_{n}} - \sqrt{x} \right| & < & \sqrt{\epsilon_{0} + x} - \sqrt{x} \\
  \end{IEEEeqnarray*}
  We can use the same strategy as last time. We let any \(\epsilon\) given to us
  be equal to \(\sqrt{\epsilon_{0} + x} - \sqrt{x}\) and solve for that particular
  \(\epsilon_{0}\). Then, we can find a \(N_{0}\) that satisfies the definition of
  convergence for \(x_{n}\). We set \(N = N_{0}\) and we have an algorithm for finding
  a \(N\) for any given \(\epsilon\). 
  
\end{enumerate}


\textbf{Exercise 2.3.5} If \(x_{n}\) and \(y_{n}\) are both convergent, then there
must exist a \(N_{1} , N_{2} \in \mathbb{N}\) such that for all \(n_{1} \geq N_{2}\)
and for all \(n_{2} \geq N_{2}\), \(\left| a_{n} - L \right| < \epsilon_{1}\) and
\(\left| b_{n} - L \right| < \epsilon_{2}\) for any \(\epsilon > 0\) since \(a_{n}\)
and \(b_{n}\) both converge to the same limit. In the case of \(z_{n}\), we choose
\(N = \max(N_{1}, N_{2})\). Then for all \(n \geq N\), we know that
\(\left|z_{n} - L \right| < \epsilon\). Thus \(z_{n}\) must converge. 

Now we must prove the converse. We start with the fact that \(z_{n}\) is convergent.
Then by definition, for any \(\epsilon > 0\), there exists \(N \in \mathbb{N}\) such
that for any \(n \in \mathbb{N}\) greater than or equal to \(N\),
\(z_{n} - L < \epsilon\), where \(L\) is the limit that \(z_{n}\) converges to. If
we split \(z_{n}\) into \(x_{n}\) and \(y_{n}\), we can use the same \(N\) when
given an \(\epsilon\) for \(x_{n}\) or \(y_{n}\). 


\textbf{Exercise 2.3.7}
\begin{enumerate}[(a)]
\item Let \(x_{n} = n\) and \(y_{n} = -n\). Both \(x_{n}\) and \(y_{n}\) diverge. However,
  \(x_{n} + y_{n}\) converge to zero.

\item Impossible by Algebraic Limit Theorem ii.
  
\item Let \(b_{n} = \frac{1}{n}\) with \(b_{n} \neq 0\) for all \(n \in \mathbb{N}\). However,
  \((1 / b_{n})\) diverges.
  
\item Impossible. By Theorem 2.3.2, \(a_{n}\) must be a divergent sequence, so then \((a_{n} - b_{n}\)
  is a sequence that diverges.
  
\item Let \(a_{n} = \frac{1}{n}\) and \(b_{n} = n\). \(a_{n}\) converges to 0 and \(b_{n}\)
  diverges. \((a_{n} b_{n})\) converges to 1.

\end{enumerate}

\end{document}
