\documentclass{article}
\usepackage{enumerate}
\usepackage[retainorgcmds]{IEEEtrantools}
\begin{document}
\textbf{6.2.3}
\begin{enumerate}[(a)]
\item In the case of \(g_{n}\), for \(c \in (0, 1)\)
  \begin{IEEEeqnarray*}{r C l"s}
    \lim g_{n} & = & \lim \frac{c}{1 + c^{n}} \\
    & = & \frac{\lim c}{\lim \left(1 + c^{n} \right)} \\
    & = & \frac{\lim c}{\lim 1 + \lim c^{n}} \\
    & = & c & Since \(0 < c < 1\), \(\lim c = \infty\), leaving only 1 in the denominator
  \end{IEEEeqnarray*}
  For \(c = 1\),
  \begin{IEEEeqnarray*}{r C l}
    g_{n}(1) & = & \frac{1}{1 + 1^{n}} \\
    & = & \frac{1}{2}
  \end{IEEEeqnarray*}
  So \(g_{n}\) converges to the function
  \[g(x) =
  \begin{cases}
    x & 0 \leq x < 1 \\
    \frac{1}{2} & x = 1 \\
    0 & x \geq 1
  \end{cases}\]

  In the case of \(h_{n}\),
  \[h(x) =
  \begin{cases}
    0 & x = 0 \\
    1 & x > 0
    \]
  
\item The convergence of \(g_{n}\) cannot be uniform on \([0, \infty]\) because
  the limit is not uniform even though \(g_{n}\) is. The same holds for \(h_{n}\)

\item \(g_{n}\) is uniform over the interval \((0, 1)\) and \(h_{n}\) is uniform
  on the interval \((1, \infty)\)
\end{enumerate}

  \textbf{6.2.6}
  \begin{enumerate}
  \item
  \item
  \item All of the functions in the sequence of functions \(f_{n}(x) = x^{n}\)
    have 0 discontinuities over the interval \([0, 1]\), yet its pointwise convergence,
    \[f(x) =
    \begin{cases}
      0 & 0 \leq x < 1 \\
      1 & x = 1\]
    \end{enumerate}
    has one discontinuity. 
\textbf{6.3.2}
\begin{enumerate}
  \item The pointwise limit of \((h_{n})\) is \(h(x) = \left| x \right| \). 
  \end{enumerate}
\textbf{6.4.3}
\textbf{6.5.2}
\begin{enumerate}[(a)]
\item Let \(a_{n} = (-1)^{n} / n! \). 
\end{enumerate}

\textbf{6.5.4}
\end{document}
