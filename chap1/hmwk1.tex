\documentclass{article}
\usepackage[retainorgcmds]{IEEEtrantools}
\usepackage{amsmath}
\usepackage{amssymb}
\usepackage{enumerate}
\author{Daniel Xu}
\title{MATH 215 HMWK}
\begin{document}
\maketitle
\begin{enumerate}
% Exercise 1
\item
  \begin{enumerate}
  \item We assume that \(\sqrt{3}\) is rational. By definition, it can be
    written in the form \(p / q\), where \(p\) and \(q\) are natural numbers.
    Let \(p / q\) be fully reduced.
    \begin{IEEEeqnarray*}{r C l"s}
      \sqrt{3} & = & \frac{p}{q} \\
      \sqrt{3} \cdot q & = & p \\
      \left( \sqrt{3} \cdot q \right)^{2} & = & p^{2} \\
      3  q^{2} & = & p^{2} & \(p\) must contain the factor 3. Let \(p = 3k, k \in \mathbb{N}\) \\
      3  p^{2} & = & (3k)^{2} \\
      3  p^{2} & = & 9k^{2} \\
      p^{2} & = & 3k^{2}
    \end{IEEEeqnarray*}
    Since \(3 \, | \, q^{2}\), then \(3 \, | \, q\). Also, \(3 \, | \, p\). This means that \(p / q\)
    can be reduced, but this contradicts our original assumption that
    \(p / q\) was fully reduced. Therefore, \(\sqrt{3}\) is not rational.
    A similar argument does work to show that \(\sqrt{6}\) is irrational.

  \item
      \begin{IEEEeqnarray*}{r C l"s}
      \sqrt{4} & = & \frac{p}{q} \\
      \sqrt{4} \cdot q & = & p \\
      \left( \sqrt{4} \cdot q \right)^{2} & = & p^{2} \\
      4  q^{2} & = & p^{2} & \(p\) can contain the factor 2. 
      \end{IEEEeqnarray*}
      The proof breaks down when we try to claim that \(p = 4k\), where
      \(k \in \mathbb{N}\). It might be possible that \(p = 2k\) since 4
      is a perfect square. 
    
  \end{enumerate}
\item We assume that \(r\) is rational. Then, it can be written as \(p / q\), where
  \(p\) and \(q\) are natural numbers. 
  \begin{IEEEeqnarray*}{r C l}
    2^{r} & = & 3 \\
    2^{p / q} & = & 3 \\
    \left(2 ^{p / q} \right)^{q} & = & 3^{q} \\
    2^{p} & = & 3^{q}
  \end{IEEEeqnarray*}
  Since \(p\) and \(q\) are natural numbers, then \(2^{p}\) must contain a factor of 3
  and \(3^{q}\) must contain a factor of 2. However, \(2^{p}\) does not contain 3 and
  \(3^{q}\) does not contain 2, so we can conclude that \(r\) is not a rational number.
\item
  \begin{enumerate}[(a)]
  \item false. When
    \begin{IEEEeqnarray*}{r C l}
      A_{1} & = & \mathbb{N} = \{1, 2, 3, \dots \} \\
      A_{2} & = & \{2, 3, 4, \ldots \} \\
      A_{3} & = & \{3, 4, 5, \ldots \} \\
      \ldots \\
      A_{n} & = & \{n, n + 1, n + 2, \ldots \}
    \end{IEEEeqnarray*}
    Then \[\bigcap_{n = 1}^{\infty} A_{n} = \emptyset \], which is clearly not infinite. 
  \item true
  \item false. Let \(A = \{1\}\), \(B = \emptyset\), and \(C = \{1, 2\}\). Then
    \begin{IEEEeqnarray*}{r C l}
      A \cap (B \cup C) & = & \{1\} \\
      (A \cap B) \cup C & = & \{1, 2\} \\
      \{1\} \neq \{1, 2\}
    \end{IEEEeqnarray*}
    
  \item true
  \item true
  \end{enumerate}
\item Let \(A_{1}\) contain 1 and every multiple of 2. Then, let \(A_{2}\) contain every multiple
  of 3 not divisible by 2. Let \(A_{3}\) contain every multiple of 5 not divisible by 2 or 3. Let
  \(A_{k}\) contain every multiple of the \(k\)th prime number not divisible by the previous \(k - 1\)
  prime numbers. Since there are an infinite number of prime numbers, there are an infinite number
  of sets. Also, there are an infinite number of multiples of prime numbers, so each set will have
  an infinite number of elements. None of the infinites sets will intersect since all of their elements
  are relatively prime to each other. 

\setcounter{enumi}{9}
\item
  \begin{enumerate}
  \item false. Let \(a = 1\), \(b = -2\), and \(c = 8\). \(1 < -2 + 8\), yet \(1 \nless -2\).
  \item false. See above.
  \item false. See above.
  \end{enumerate}

\setcounter{enumi}{11}
\item
  \begin{enumerate}
  \item Base Case: \(y_{1} = 6 > -6\).

    Inductive Step: First, we assume that \(y_{n} > 6\). Now we must show that if \(y_{k} > -6\),
    then \(y_{k+1} > 6\).
    \begin{IEEEeqnarray*}{r C l}
      y_{k} & > & -6 \\
      2y_{k} & > & -12 \\
      2y_{k} - 6 & > & -18 \\
      \frac{2y_{k} - 6}{3} & > & -6 \\
      y_{k+1} & > & -6
    \end{IEEEeqnarray*}
  \item Base case:
    \begin{IEEEeqnarray*}{r C l}
      y_{1} & \stackrel{=}{>} & y_{2} \\
      6 & \stackrel{=}{>} & \frac{2 \cdot 6 - 6}{3} \\
      6 & \stackrel{=}{>} & \frac{6}{3} \\
      6 & > & 2
    \end{IEEEeqnarray*}

    Inductive Step: We assume that \(y_{n} > y_{n+1}\). Now, we must show that if \(y_{k} > y_{k+1}\),
    then \(y_{k+1} > y_{k+2}\). 
    \begin{IEEEeqnarray*}{r C l}
      y_{k} & > & y_{k+1} \\
      2 y_{k} & > & 2 y_{k+1} \\
      2 y_{k} - 6 & > & 2 y_{k+1} - 6 \\
      \frac{2 y_{k} - 6}{3} & > & \frac{2 y_{k+1} - 6}{3} \\
      y_{k+1} & > & y_{k+2}
    \end{IEEEeqnarray*}
    
  \end{enumerate}
    
\end{enumerate}
\end{document}
