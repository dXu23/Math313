\documentclass{article}
\usepackage[retainorgcmds]{IEEEtrantools}
\usepackage{amsmath}
\usepackage{amssymb}
\usepackage{enumerate}
\author{Daniel Xu}
\title{Chapter 1}
\begin{document}
\maketitle
\begin{enumerate}
% Exercise 1.2.1
\item
  \begin{enumerate}
  \item We assume that \(\sqrt{3}\) is rational. By definition, it can be
    written in the form \(p / q\), where \(p\) and \(q\) are natural numbers.
    Let \(p / q\) be fully reduced.
    \begin{IEEEeqnarray*}{r C l"s}
      \sqrt{3} & = & \frac{p}{q} \\
      \sqrt{3} \cdot q & = & p \\
      \left( \sqrt{3} \cdot q \right)^{2} & = & p^{2} \\
      3  q^{2} & = & p^{2} & \(p\) must contain the factor 3. Let \(p = 3k, k \in \mathbb{N}\) \\
      3  p^{2} & = & (3k)^{2} \\
      3  p^{2} & = & 9k^{2} \\
      p^{2} & = & 3k^{2}
    \end{IEEEeqnarray*}
    Since \(3 \, | \, q^{2}\), then \(3 \, | \, q\). Also, \(3 \, | \, p\). This means that \(p / q\)
    can be reduced, but this contradicts our original assumption that
    \(p / q\) was fully reduced. Therefore, \(\sqrt{3}\) is not rational.
    A similar argument does work to show that \(\sqrt{6}\) is irrational.

  \item
      \begin{IEEEeqnarray*}{r C l"s}
      \sqrt{4} & = & \frac{p}{q} \\
      \sqrt{4} \cdot q & = & p \\
      \left( \sqrt{4} \cdot q \right)^{2} & = & p^{2} \\
      4  q^{2} & = & p^{2} & \(p\) can contain the factor 2. 
      \end{IEEEeqnarray*}
      The proof breaks down when we try to claim that \(p = 4k\), where
      \(k \in \mathbb{N}\). It might be possible that \(p = 2k\) since 4
      is a perfect square. 
    
  \end{enumerate}

% Exercise 1.2.2
\item We assume that \(r\) is rational. Then, it can be written as \(p / q\), where
  \(p\) and \(q\) are natural numbers. 
  \begin{IEEEeqnarray*}{r C l}
    2^{r} & = & 3 \\
    2^{p / q} & = & 3 \\
    \left(2 ^{p / q} \right)^{q} & = & 3^{q} \\
    2^{p} & = & 3^{q}
  \end{IEEEeqnarray*}
  Since \(p\) and \(q\) are natural numbers, then \(2^{p}\) must contain a factor of 3
  and \(3^{q}\) must contain a factor of 2. However, \(2^{p}\) does not contain 3 and
  \(3^{q}\) does not contain 2, so we can conclude that \(r\) is not a rational number.

% Exercise 1.2.3
\item
  \begin{enumerate}[(a)]
  \item false. When
    \begin{IEEEeqnarray*}{r C l}
      A_{1} & = & \mathbb{N} = \{1, 2, 3, \dots \} \\
      A_{2} & = & \{2, 3, 4, \ldots \} \\
      A_{3} & = & \{3, 4, 5, \ldots \} \\
      \ldots \\
      A_{n} & = & \{n, n + 1, n + 2, \ldots \}
    \end{IEEEeqnarray*}
    Then \[\bigcap_{n = 1}^{\infty} A_{n} = \emptyset \], which is clearly not infinite. 
  \item true
  \item false. Let \(A = \{1\}\), \(B = \emptyset\), and \(C = \{1, 2\}\). Then
    \begin{IEEEeqnarray*}{r C l}
      A \cap (B \cup C) & = & \{1\} \\
      (A \cap B) \cup C & = & \{1, 2\} \\
      \{1\} \neq \{1, 2\}
    \end{IEEEeqnarray*}
    
  \item true
  \item true
  \end{enumerate}
% Exercise 1.2.4
\item Let \(A_{1}\) contain 1 and every multiple of 2. Then, let \(A_{2}\) contain every multiple
  of 3 not divisible by 2. Let \(A_{3}\) contain every multiple of 5 not divisible by 2 or 3. Let
  \(A_{k}\) contain every multiple of the \(k\)th prime number not divisible by the previous \(k - 1\)
  prime numbers. Since there are an infinite number of prime numbers, there are an infinite number
  of sets. Also, there are an infinite number of multiples of prime numbers, so each set will have
  an infinite number of elements. None of the infinites sets will intersect since all of their elements
  are relatively prime to each other. 


% Exercise 1.2.5
\item
\begin{enumerate}[(a)]
\item If \(x \in (A \cup B)^{c}\), then \(x \notin (A \cup B)\). This means that
\(x \notin A\) or \(x \notin B\), which is what \(A^{c} \cup B^{c}\) means. 
\item If \(x \in A^{c} \cap B^{c}\), then \(x \notin A\) or \(x \notin B\). This means
that \(x \notin (A \cup B)\), which means \(x \in (A \cup B)^{c}\). 
\item Since \( (A \cup B)^{c} \subset A^{c} \cup B^{c} \) and \( A^{c} \cup B^{c} \subset (A \cup B)^{c}\),
	it is implied that \((A \cup B)^{c} = A^{c} \cap B^{c}\). 
\end{enumerate}

% Exercise 1.2.6
\item
\begin{enumerate}
\item We consider the case when \(a > 0\) and \(b > 0\). Then \(a + b > 0\), so
\(\left| a + b \right| = a + b\). Since \(a > 0\), then \(\left| a \right| = a\).
This is similarly true for \(b\). So \(\left| a \right| + \left| b \right| = a + b\).
By the transitive property, \(\left| a + b \right| = \left| a \right| + \left| b \right|\),
   which verifies the triangle inequality for all \(a > 0\) and \(b > 0\). 


We consider the case when \(a < 0\) and \(b < 0\). Then \(a + b < 0\). Thus,
   \(\left| a + b \right| = -(a + b)\). Since \(a < 0\), \(\left| a \right| = -a\).
   This is similarly true for \(b\). So
   \(\left| a \right| + \left| b \right| = -a + (-b) = - (a + b)\). 
   By the transitive property, \(\left| a + b \right| = \left| a \right| + \left| b \right|\),
   verifying the triangle inequality for \(a < 0\) and \(b < 0\).
   
	\item For all numbers, it is true that the square of the absolute value of 
	the number is greater or equal than the square of the number. Therefore
	\begin{IEEEeqnarray*}{r C l}
	  a^{2} + 2ab + b^{2} & \leq & |a|^{2} + 2 | a | | b| + | b |^{2} \\
	  (a + b)^{2} & \leq & \left(| a | + | b | \right)^{2}
	\end{IEEEeqnarray*}

	\item 
	\begin{IEEEeqnarray*}{r C l"s}
		| a - b | & = & | a - c + c - d + d - b | \\
					  & \leq & | a - c | + | c - d + d - b | & By Triangle Inequality \\
					  & \leq & | a - c | + | c - d | + | d - b| & By Triangle Inequality
	\end{IEEEeqnarray*}

	\item
	\begin{IEEEeqnarray*}{r C l"s}
	| | a | - | b | | & = & | | a - b + b | - | b | | & By identity property \\
					  & \leq & | | a - b | + | b | - | b | | & By triangle inequality \\ 
					  & = & | a - b |
	\end{IEEEeqnarray*}

\end{enumerate}

% Exercise 1.2.7
\item 
\begin{enumerate}
	\item Since \(A = [0, 2]\) and \(B = [1, 4]\), then \(f(A) = [0, 4]\) and \(f(B) = [1, 16]\).
	\(A \cap B = [1, 2]\) and \(A \cup B = [0, 4]\), so \(f \left( A \cap B \right) = [1, 4]\)
	and \(f \left(A \cup B \right) = [0, 16]\). \(f(A) \cup f(B) = [0, 16]\), so 
	\(f \left(A \cup B \right) = f(A) \cup f(B)\). Likewise, \(f(A) \cap f(B) = [1, 4]\), so
	\(f \left(A \cap B \right) = f(A) \cap f(B)\). 

	\item Let \(A = [0, 2]\) and \(B = [-3, 0]\). Then \(A \cap B = [0, 0]\), \(f(A) = [0, 4]\),
	and \(f(B) = [0, 9]\). However, \(f(A \cap B) = [0, 0]\) and \(f(A) \cap f(B) = [0, 4]\). 
	Evidently, \(f(A \cap B) \neq f(A) \cap f(B)\). 

	\item In the case of \(g(A \cap B)\), it is true by definition that 
	\(\forall g(x) \in g(A \cap B) \ldotp x \in A \cap B\). If \(x \in A \cap B\),
	then \(x \in A\) and \(x \in B\). This implies that \(g(x) \in g(A)\) and \(g(x) \in g(B)\)
	by definition. This is exactly what \(g(A) \cap g(B)\) is. 

	\item We conjecture that for any arbitrary function \(g : \mathbb{R} \rightarrow \mathbb{R}\),
	it is always true that \(g(A \cup B) \subset g(A) \cup g(B)\). 

	\(\forall g(x) \in g(A \cup B) \ldotp x \in A \cup B\). If \(x \in A \cup B\), then \(x \in A\)
	or \(x \in B\). In the case where \(x \in A\) but \(x \notin B\), this implies that \(g(x) \in A\),
	which means that \(g(x) \in g(A \cup B)\). In the case where \(x \in B\) but \(x \notin A\), 
	\(g(x) \in B\), which means that \(g(x) \in g(A \cup B)\). This is evidently also true when
	\(x \in A\) and \(x \in B\). 
\end{enumerate}

\item 
\begin{enumerate}[(a)]
	\item \(f(x) = 2x\) is an example of such a function. Let \(a_{1} \in A\) and \(a_{2} \in A\), with
	\(a_{1} \neq a_{2}\). Then \(2a_{1} \neq 2a_{2}\). However, if we choose \(3 \in \mathbb{N}\), there
	is no natural number \(a \in \mathbb{N}\) such that \(f(a) = 3\). 

	\item Let \(f(x) = | x - 2 | + 1\). Then for any \(b \in \mathbb{N}\), it is always true that 
	\(a = b + 1\) is in \(\mathbb{N}\) and \(f(a) = b\). However, \(f(1) = 2\) and \(f(3) = 2\) even
	though \(1 \neq 3\). 

	\item
	\begin{IEEEeqnarray*}{r C l}
	f(x) = \begin{cases}
			 \frac{x - 1}{2} \textrm{ if } x \textrm{ is odd} \\
			 - \frac{x}{2} \textrm{ if } x \textrm{ is even} \\
	       \end{cases}
	\end{IEEEeqnarray*}
	
\end{enumerate}

% Exercise 1.2.9
\item
\begin{enumerate}[(a)]
	\item \(f^{-1} (A) = \{-2, 2\}\) and \(f^{-1} (B) = \{-1, 1\}\). \(f^{-1} (A \cap B) = [-1, 1]\)
	and \(f^{-1} (A) \cap f^{-1} (B) = \{-1, 1\}\), so \(f^{-1} (A \cap B) = f^{-1} (A) \cap f^{-1} (B)\).
	\(f^{-1} (A \cup B) = [-2, 2]\) and \(f^{-1} (A) \cup f^{-1} (B) = [-2, 2]\), so
	\(f^{-1} (A \cup B) = f^{-1} (A) \cup f^{-1} (B)\).

	\item \(x \in g^{-1} (A \cap B)\) implies that \(g(x) \in A \cap B\), so \(g(x) \in A\)
	and \(g(x) \in B\). From this, we can conclude that \(x \in g^{-1} (A)\) and \(x \in g^{-1} (B)\),
	which is equivalent to \(g^{-1} (A) \cap g^{-1} (B)\). Thus, we have shown that
	\(x \in g^{-1} (A \cap B) \implies x \in g^{-1} (A) \cap g^{-1} (B)\), meaning
	\(g^{-1} (A \cap B) \subset g^{-1} (A) \cap g^{-1} (B)\). We must show that 
    \(g^{-1} (A) \cap g^{-1} (B) \subset g^{-1} (A \cap B)\)

	If \(x \in g^{-1} (A) \cap g^{-1} (B)\), then \(x \in g^{-1} (A)\) and \(x \in g^{-1} (B)\). 
	This implies that \(g(x) \in A\) and \(g(x) \in B\), meaning that \(g(x) \in A \cap B\), so
	\(x \in g^{-1} (A \cap B)\). Thus, we have shown that 
	\(g^{-1} (A) \cap g^{-1} (B) \subset g^{-1} (A \cap B)\).

	Thus, since \(g^{-1} (A) \cap g^{-1} (B) \subset g^{-1} (A \cap B)\) and
	\(g^{-1} (A) \cap g^{-1} (B) \subset g^{-1} (A \cap B)\), we can conclude
	that \(g^{-1} (A \cap B) = g^{-1} (A) \cap g^{-1} (B)\).
	
	\(x \in g^{-1} (A \cup B)\) implies that \(g(x) \in A \cup B\), meaning that \(g(x) \in A\)
	or \(g(x) \in B\). This is equivalent to \(x \in g^{-1} (A)\) or \(x \in g^{-1} (B)\), so
	\(x \in g^{-1} (A) \cup g^{-1} (B)\). This, we have shown that 
	\(g^{-1} (A \cup B) \subset g^{-1} \cup g^{-1} (B)\). 

	\(x \in g^{-1} (A) \cup g^{-1} (B)\) means that \(x \in g^{-1} (A)\) or \(x \in g^{-1} (B)\). 
	This implies that \(g(x) \in A\) or \(g(x) \in B\), which is equivalent to \(g(x) \in (A \cup B)\). 
	From this, we conclude that \(x \in g^{-1} (A \cup B)\). Thus, we have shown that
	\(g^{-1} (A) \cup g^{-1} (B) \subset g^{-1} (A \cup B)\).

	Since \(g^{-1} (A) \cup g^{-1} (B) \subset g^{-1} (A \cup B)\) and
	\(g^{-1} (A \cup B) \subset g^{-1} \cup g^{-1} (B)\), we can conclude that
	\(g^{-1} (A \cup B) = g^{-1} (A) \cup g^{-1} (B)\).  


\end{enumerate}
	
% Exercise 1.2.10
\item
  \begin{enumerate}
  \item false. Let \(a = 1\), \(b = -2\), and \(\epsilon = 8\). \(1 < -2 + 8\), yet \(1 \nless -2\).
  \item false. Let \(a = 1\), \(b = -2\), and \(\epsilon = 8\). \(1 < -2 + 8\), yet \(1 \nless -2\).
  \item false. Let \(a = 1\), \(b = -2\), and \(\epsilon = 8\). \(1 < -2 + 8\), yet \(1 \nleq -2\).
  \end{enumerate}

  % Exercise 1.2.11
  \item 
  \begin{enumerate}[(a)]
	  \item There exists real number satisfying \(a < b\), such that for all \(n \in \mathbb{N}\)
	such that \(a + 1 / n > b\). 

	\item For all real numbers \(x > 0\) there exists \(n \in \mathbb{N}\) such that \(x > 1 / n\). 

	\item There exists two real distinct real numbers for which there is no rational number between
	them. 
  \end{enumerate}

% Exercise 1.2.12
\item
  \begin{enumerate}
  \item Base Case: \(y_{1} = 6 > -6\).

    Inductive Step: First, we assume that \(y_{n} > 6\). Now we must show that if \(y_{k} > -6\),
    then \(y_{k+1} > 6\).
    \begin{IEEEeqnarray*}{r C l}
      y_{k} & > & -6 \\
      2y_{k} & > & -12 \\
      2y_{k} - 6 & > & -18 \\
      \frac{2y_{k} - 6}{3} & > & -6 \\
      y_{k+1} & > & -6
    \end{IEEEeqnarray*}
  \item Base case:
    \begin{IEEEeqnarray*}{r C l}
      y_{1} & \stackrel{?}{>} & y_{2} \\
      6 & \stackrel{?}{>} & \frac{2 \cdot 6 - 6}{3} \\
      6 & \stackrel{?}{>} & \frac{6}{3} \\
      6 & > & 2
    \end{IEEEeqnarray*}

    Inductive Step: We assume that \(y_{n} > y_{n+1}\). Now, we must show that if \(y_{k} > y_{k+1}\),
    then \(y_{k+1} > y_{k+2}\). 
    \begin{IEEEeqnarray*}{r C l}
      y_{k} & > & y_{k+1} \\
      2 y_{k} & > & 2 y_{k+1} \\
      2 y_{k} - 6 & > & 2 y_{k+1} - 6 \\
      \frac{2 y_{k} - 6}{3} & > & \frac{2 y_{k+1} - 6}{3} \\
      y_{k+1} & > & y_{k+2}
    \end{IEEEeqnarray*}
    
  \end{enumerate}

  % Exercise 1.2.13
  \item
  \begin{enumerate}[(a)]
  	\item We consider the base case. Evidently, \(A_{1}^{c} = A_{1}^{c}\)
	by the reflexive property, so we are already done with the base case.

	Now we move on to the inductive step. Now we assume that the statement
	is true for \(n\) and we have show that if it is true for \(n = k\), 
	then it is true for \(n = k + 1\). 

	\begin{IEEEeqnarray*}{r C l"s}
	(A_{1} \cup A_{2} \cup \ldots \cup A_{k})^{c} & = & A_{1}^{c} \cap A_{2}^{c} \cap \ldots \cap A_{k}^{c} \\
	(A_{1} \cup A_{2} \cup \ldots \cup A_{k})^{c} \cap A_{k + 1}^{c} & = & A_{1}^{c} \cap A_{2}^{c} \cap \ldots \cap A_{k}^{c} \cap_{A_{k + 1}^{c}\\
	(A_{1} \cup A_{2} \cup \ldots \cup A_{k} \cap A_{k + 1})^{c} & = & A_{1}^{c} \cap A_{2}^{c} \cap \ldots \cap A_{k}^{c} \cap_{A_{k + 1}^{c} & By DeMorgan's Law \\
	\end{IEEEeqnarray*}

	Therefore, we can conclude that 
	\((A_{1} \cup A_{2} \cup \ldots \cup A_{n})^{c} & = & A_{1}^{c} \cap A_{2}^{c} \cap \ldots \cap A_{n}^{c}\)

	\item Let
    \begin{IEEEeqnarray*}{r C l}
      B_{1} & = & \mathbb{N} = \{1, 2, 3, \dots \} \\
      B_{2} & = & \{2, 3, 4, \ldots \} \\
      B_{3} & = & \{3, 4, 5, \ldots \} \\
      \ldots \\
      B_{n} & = & \{n, n + 1, n + 2, \ldots \}
    \end{IEEEeqnarray*}
	Then \(\bigcap_{i = 1}^{n} B_{i} = B_{n}\), but \(\bigcap_{i = 1}^{\infty} B_{i} = \emptyset\).


	\item
	
	\item 
  \end{enumerate}
    
\end{enumerate}
\end{document}
